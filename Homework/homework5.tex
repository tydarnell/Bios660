\documentclass{article}
\usepackage[utf8]{inputenc}
\usepackage[english]{babel}
\usepackage{graphicx}
\usepackage{enumerate}
\usepackage{float}
\graphicspath{ {} }
\usepackage{mathtools}
\usepackage{amsmath, amsthm, amssymb, amsfonts}
\usepackage{caption}
\usepackage{fancyhdr}
\pagestyle{fancy}
\fancyhf{}
\rhead{Ty Darnell}
\lhead{Homework 5}

% For derivatives
\newcommand{\deriv}[1]{\frac{\mathrm{d}}{\mathrm{d}x} (#1)}

% For partial derivatives
\newcommand{\pderiv}[2]{\frac{\partial}{\partial #1} (#2)}

% Integral dx
\newcommand{\dx}{\mathrm{d}x}
\begin{document}
\begin{flushleft}
\chead{Problems 1-5}
\section*{Problem 1}
\begin{enumerate}[(a)]
\item $P(\text{Doubles})=\dfrac{1}{6}*\dfrac{1}{6}*6=\dfrac{1}{6}$
\item $P(D) = \text{Doubles}=\dfrac{1}{6}$ \quad $P(A)=\text{Sum of 4 or less}=\dfrac{6}{36}=\dfrac{1}{6}$\medbreak
$P(D|A)=\dfrac{P(A|D)P(D)}{P(A)}$\medbreak
$P(D|A)=\dfrac{(1/3)*(1/6)}{(1/6)}=\dfrac{1}{3}$
\item $P(\text{at least one 6})=1-P(\text{no 6's})=1-\dfrac{5}{6}*\dfrac{5}{6}=1-\dfrac{25}{36}=\dfrac{11}{36}$
\item $P(B)=\text{Different numbers}=1-P(\text{Doubles}) =\dfrac{5}{6}$ \quad $P(A)=\text{At least one six}=\dfrac{11}{36}$\medbreak
$P(A|B)= \dfrac{P(B|A)*P(A)}{P(B)}=\dfrac{(10/11)*(11/36)}{(5/6)}=\dfrac{1}{3}$
\end{enumerate}

\section*{Problem 2}
The probability of drawing a white ball from jar 1 is $P(W_1)$. We can use the law of total probability to find the probability of drawing a white ball from the $k_{th}$ jar since we are either adding a black or white ball to every jar after the first 1.
\begin{align*}
P(W_1)&=\dfrac{m}{m+n}\\
P(W_k)&=P(W_{k-1})P(W_k|W_{k-1})+P({W_{k-1}}^c)P(W_k|{W_{k-1}}^c)\\
&=\dfrac{m}{m+n}\dfrac{m+1}{m+n+1}+\dfrac{n}{m+n}\dfrac{m}{m+n+1}\\
&=\dfrac{m(m+1)+nm}{(m+n)(m+n+1)}\\
&=\dfrac{m(m+n+1)}{(m+n)(m+n+1)}\\
&=\dfrac{m}{m+n}\\
&\text{Thus } P(W_1)=P(W_k)
\end{align*}
\pagebreak
\section*{Problem 3}
\begin{align*}
\text{We want to show:}&\\
P(A|B)=P(C|B)P(A|B\cap C)&+P(C^c|B)P(A|B \cap C^c)\\
\text{Using the definition } &\text{of conditional probability:}\\
P(C|B)&=\dfrac{P(C\cap B)}{P(B)}\\
P(A|B\cap C)&=\dfrac{P(A\cap B\cap C)}{P(B\cap C)}\\
P(C^c|B)&=\dfrac{P(C^c\cap B)}{P(B)}\\
P(A|B\cap C^c)&=\dfrac{P(A\cap B\cap C^c)}{P(B\cap C^c)}\\
\text{Putting this}&\text{ together we can write}\\
P(C|B)P(A|B\cap C)&+P(C^c|B)P(A|B \cap C^c) \text{ as:}\\
\dfrac{P(C\cap B)}{P(B)}\dfrac{P(A\cap B\cap C)}{P(B\cap C)}&+
\dfrac{P(C^c\cap B)}{P(B)}\dfrac{P(A\cap B\cap C^c)}{P(B\cap C^c)}\\
=\dfrac{P(A\cap B\cap C)}{P(B)}&+\dfrac{P(A\cap B\cap C^c)}{P(B)}\\
=&\dfrac{P(A\cap B)}{P(B)}\\
=&P(A|B)
\end{align*}
\section*{Problem 4}
\begin{enumerate}[(a)]
\item
\begin{align*}
\text{We want to prove: } &P(A\cap B^c)=P(A)P(B^c)\\
\text{Given } P(A\cap B)&=P(A)+P(B)\\
\text{We can write } &A \text{ as:}\\
A=&(A\cap B)\cup (A\cap B^c)\\
\text{Since the union is}&\text{ disjoint we can write:}\\
P(A)&=P(A\cap B)+P(A\cap B^c)\\
P(A)&=P(A)P(B)+P(A\cap B^c)\\
P(A\cap B^c)&=P(A)-P(A)P(B)\\
P(A\cap B^c)&=P(A)(1-P(B))\\
P(A\cap B^c)&=P(A)P(B^c)\\
\text{Therefore } A \text{ and } B^c & \text{ are independent}
\end{align*}
\item
\begin{align*}
\text{We want to prove: }& P(A^c\cap B^c)=P(A^c)P(B^c)\\
\text{We can write } &B^c \text{ as:}\\
B^c=&(B^c\cap A)\cup (B^c\cap A^c)\\
\text{Since the union is}&\text{ disjoint we can write:}\\
P(B^c)&=P(B^c\cap A)+P(B^c\cap A^c)\\
\text{Using part a}&\text{ we can write:}\\
P(B^c)&=P(A)P(B^c)+P(B^c\cap A^c)\\
P(B^c\cap A^c)&=P(B^c)-P(A)P(B^c)\\
P(B^c\cap A^c)&=P(B^c)(1-P(A))\\
P(B^c\cap A^c)&=P(B^c)P(A^c)\\
\text{Therefore } A^c \text{ and } B^c & \text{ are independent}
\end{align*}
\end{enumerate}
\section*{Problem 5}
\textbf{Game 1:} P(win)=P(tie)=(.4)(.5)\\
P(loss)=(.6)\medbreak
\textbf{Game 2:} P(win)=P(tie)=(.7)(.5)\\
P(loss)=(.3)\medbreak
\textbf{PMF:}\\
$P(0)=(.6)(.3)=.18$\\
$P(1)=(.4)(.5)(.3)+(.6)(.7)(.5)=.27$\\
$P(2)=(.4)(.5)(.7)(.5)+(.4)(.5)(.3)+(.6)(.7)(.5)=.34$\\
$P(3)=(.4)(.5)(.7)(.5)(2)=.14$\\
$P(4)=(.4)(.5)(.7)(.5)=.07$
\pagebreak
\section*{Problem 6}
\chead{Problems 6-10}
\begin{enumerate}[(a)]
\item
P(Harry Wins)= $\sum_{1}^{10}(.3)^n$\medbreak $=(.3)+(.3)^2+.(3)^3+(.3)^4+(.3)^5+.(3)^6+(.3)^7+(.3)^8+.(3)^9+(.3)^{10}$\\
$\approx 0.4285689$ \medbreak

There is a .3 probability Harry wins the first game plus the probability the first game is drawn times .3 probability he wins the second and so on all the way to 10 games.
\item
$P(1)=.4+.3$\\
$P(2)=(.3)(.4)+(.3)(.3)$\\
$P(3)=.3^2(.4+.3)$\\
$P(4)=.3^3(.4+.3)$\\
$P(5)=.3^4(.4+.3)$\\
$P(6)=.3^5(.4+.3)$\\
$P(7)=.3^6(.4+.3)$\\
$P(8)=.3^7(.4+.3)$\\
$P(9)=.3^8(.4+.3)$\\
$P(10)=.3^9(.4+.3)+.3^{10}$\\
\end{enumerate}
\section*{Problem 7}
$P(M)=P(W)=.5$\\
$P(M\cap C)=.05$\\
$P(W \cap C)=.0025$\\
$P(C|M)=\dfrac{P(C\cap M)}{P(M)}=.05/.5=.1$\\
$P(C|W)=\dfrac{P(C\cap W)}{P(W)}=.0025/.5=.005$\\
$P(C)=P(M)P(C|M)+P(W)P(C|W)=(.5)(.1)+(.5)(.005)=.0525$\medbreak
$P(M|C)=\dfrac{P(C|M)P(M)}{P(C)}=(.1)(.5)/(.0525)=.952381$
\section*{Problem 8}
\begin{enumerate}[(a)]
\item	
$P(H)=1/5$\\
$P(H\geq 2)=1-[P(H=0)+P(H=1)]$\\
$P(H=0)=(4/5)^{10}$\\
$P(H=1)=(4/5)^9(1/5)(10)$\\
$P(H\geq 2)=1-[(4/5)^{10}+(4/5)^9(1/5)(10)]\approx .6241904$
\item
$P(H \geq 1)=1-P(H=0)\approx .8926258$\medbreak
$P(H\geq 2|H \geq 1)=\dfrac{P(H\geq 2 \cap H \geq 1)}{P(H\geq 1)}$\medbreak
$=\dfrac{P(H\geq 2)}{P(H \geq 1)}\approx.6992744$
\end{enumerate}
\section*{Problem 9}
\begin{enumerate}[(a)]
\item 
We want to prove $P(B)=1 \Longrightarrow P(A|B)=P(A) \ \forall \ A$\\
Assume $P(B)=1$\\
By definition of conditional probability:\\ $P(A|B)=\dfrac{P(B|A)P(A)}{P(B)}$\\
Then $P(A|B)=1(P(A))/1=P(A)$
\item
We want to prove $A \subset B \Longrightarrow P(B|A)=1 \text{ and } P(A|B)=P(A)/P(B)$\\
Assume $A \subset B$\\
Then $A\cap B=A$\\
Thus $P(A\cap B)=P(A)$\\
Since $P(A|B)=P(A\cap B)/P(B)$\\
We have $P(A|B)=P(A)/P(B)$\\
$P(B|A)=\dfrac{P(A|B)P(B)}{P(A)}=\dfrac{(P(A)/P(B))P(B)}{P(A)}=P(A)/1/P(A)=1$\\
\item
We want to prove $A \cap B=\emptyset \Longrightarrow P(A|A\cup B)=\dfrac{P(A)}{P(A)+P(B)}$\\
Assume $A \cap B=\emptyset$\\
This means $P(A\cap B)=0$ and $P(A\cup B)=P(A)+P(B)$\\
From the definition of conditional probability we can write:\\ $P(A|A\cup B)=\dfrac{P(A\cap (A\cup B))}{P(A\cup B)}$\\
Since $P(A\cap (A\cup B))=P(A)$  we have:\\
$P(A|A\cup B)=\dfrac{P(A)}{P(A)+P(B)}$\\
\item
We want to prove $P(A\cap B \cap C)=P(A|B\cap C)P(B|C)P(C)$\\
We can write the left side as $P(A\cap (B\cap C))$ since intersections are associative.\\
 $P(A\cap (B\cap C))=P(B\cap C)P(A|B\cap C)$ By definition of conditional probability\\
 Using the definition of conditional probability again to rewrite $P(B\cap C)$ we have:\\
 $P(A|B\cap C)P(B|C)P(C)$\\
 Therefore $P(A\cap B \cap C)=P(A|B\cap C)P(B|C)P(C)$
\end{enumerate}
\section*{Problem 10}
\begin{enumerate}[(a)]
\item
We want to prove If $P(A)>0$ and $P(B)>0$ then if $A$ and $B$ are mutually exclusive, they cannot be independent, that is $P(A\cap B)\neq P(A)*P(B)$.\\
Assume $P(A)>0$, $P(B)>0$ and $A\cap B=\emptyset$\\
Since $P(A\cap B)=P(\emptyset)=0$\\
Then $P(A\cap B)=0$\\
Since $P(A)>0$ and $P(B)>0$ $P(A\cap B)$ cannot equal 0.\\
This creates a contradiction, thus two events with positive probabilities that are mutually exclusive cannot be independent.
\item 
Given two independent events, where $P(A)>0$ and $P(B)>0$ we want to show $A$ and $B$ cannot be mutually exclusive, that is $A\cap B$ is nonempty.
Since A and B are independent, and have probabilities greater than 0, $P(A\cap B)=P(A)P(B)>0$\\
Thus the intersection cannot be empty since $P(A\cap B)\neq 0$\\
Therefore $A$ and $B$ cannot be mutually exclusive.
\end{enumerate}
\pagebreak
\section*{Problem 11}
\chead{Problems 11-16}
$P(Correct\geq 10|Guessing)=\sum_{k=10}^{20}{20\choose k}(1/4)^k(3/4)^{20-k}=.01386442$
\section*{Problem 12}
Given a sample space $S=\left\{s_1,\dots,s_n \right\}$\\
with a probability function $P$, we define a random variable X with a range of $\chi=\left\{x_1,\dots x_m \right\}$\\
$P_X$ is defined as an induced probability function on $\chi$ such that:\\
$P_X(X=x_i)=P(\left\{s_j\in S:X(s_j)=x_i \right\})$\\
We want to prove this a legitimate probability function that satisfies the Kolmogorov Axioms.\medbreak
\textbf{Proof:}\\
Since $X$ has a finite range, $X$ is finite.\\
Therefore $\mathcal{B}$ is the set of all subsets of $\chi$.\\
If $A \in B$ then $P_X(A)=P(\cup_{x_i\in A}\left\{s_j\in S:X(s_j)=x_i \right\})\geq 0$ since we know $P$ is a probability function.\\
Thus Axiom 1 holds.\medbreak
$P_X(\chi)=P(\cup_{i=1}^{m}\left\{\cup_{s_j \in S:X(s_j)=x_i} \right\})=P(S)=1$\\
Therefore Axiom 2 holds.\medbreak
If $A_1,A_2,\dots \in \mathcal{B}$ and pairwise disjoint then:\\
\[P_X(\cup_{k=1}^{\infty}A_k)=P(\cup_{k=1}^{\infty}\left\{\cup_{\chi_i \in A_k}\left\{s_j\in S:X(s_j)=x_i \right\} \right\})\]
\[=\sum_{k=1}^{\infty}P(\cup_{\chi_i \in A_k}\left\{s_j\in S:X(s_j)=x_i \right\})=\sum_{k=1}^{\infty}P_X(A_k)
\]
Thus the third Axioms holds.\\
Therefore we have satisfied all three axioms and have defined a legitimate probability function.

\section*{Problem 13} Functions a-d are continuous, so they are all right-continuous.
\begin{enumerate}[(a)]
\item
$\lim_{x\to \infty} \tan^{-1}(x)=\dfrac{\pi}{2}$\\
$\lim_{x\to -\infty} \tan^{-1}(x)=\dfrac{-\pi}{2}$\\
Therefore $\lim_{x\to \infty}1/2 +(1/\pi)\tan^{-1}(x)=\dfrac{\pi}{2}\dfrac{1}{\pi}+1/2=1$\\
$\lim_{x\to -\infty}1/2 +(1/\pi)\tan^{-1}(x)=\dfrac{-\pi}{2}\dfrac{1}{\pi}+1/2=0$\\
\textbf{Nondecreasing:}\\
$\deriv{1/2+1/\pi \tan^{-1}(x)}=\dfrac{1}{1+x^2}>0$\\
Therefore the function satisfies all three properties and is a cdf.
\item
$\lim_{x\to \infty}(1+e^{-x})^{-1}=(1+0)^{-1}=1$\\
$\lim_{x\to -\infty}(1+e^{-x})^{-1}=\dfrac{1}{1+\infty}=0$\\
\textbf{Nondecreasing:}\\
$\deriv{(1+e^{-x})^{-1}}=e^{-x}(1+e^{-x})^{-2}>0$\\
Therefore the function satisfies all three properties and is a cdf.
\item
$\lim_{x\to \infty} e^{-e^{-x}}=e^0=1$\\
$\lim_{x\to -\infty} e^{-e^{-x}}=e^{-\infty}=0$\\
\textbf{Nondecreasing:}\\\
$\deriv{e^{-e^{-x}}}=e^{-x}e^{-e^{-x}}>0$\\
Therefore the function satisfies all three properties and is a cdf.
\item
$\lim_{x\to \infty}1-e^{-x}=1-0=1$\\
$\lim_{x\to 0}1-e^{-x}=1-1=0$\\
\textbf{Nondecreasing:}\\\
$\deriv{1-e^{-x}}=e^{-x}>0$\\
Therefore the function satisfies all three properties and is a cdf.
\item
\[F_Y(y)=\begin{cases}
\dfrac{1-\epsilon}{1+e^{-y}} \quad \text{if } y<0\\
\epsilon + \dfrac{(1-\epsilon)}{1+e^{-y}} \quad \text{if } y \geq 0
\end{cases}
\]
Where $0<\epsilon<1$\medbreak
$F_Y(y)$ is continuous except at $y=0$, where the limit=$F(0)$, thus right continuous.\\
$\lim_{y\to \infty}F_Y(y)=\epsilon+\dfrac{1-\epsilon}{1+0}=1$\\
$\lim_{y\to -\infty}F_Y(y)=\dfrac{1-\epsilon}{1+\infty}=0$\\
\textbf{Nondecreasing:}\\\
$\deriv{\dfrac{1-\epsilon}{1+e^{-y}}}=(1-\epsilon)(1+e^{-y})^{-2}e^{-y}>0$\\
$\deriv{\epsilon+\dfrac{1-\epsilon}{1+e^{-y}}}=\epsilon+(1-\epsilon)(1+e^{-y})^{-2}e^{-y}>0$\\
Therefore the function satisfies all three properties and is a cdf.
\end{enumerate}
\section*{Problem 14}
\begin{enumerate}[(a)]
\item
$F_Y$ is continous over $[1,\infty)$ so it is right continuous.\\
$\lim_{y\to 1}F_Y=1-1=0$\\
$\lim_{y\to \infty}F_Y=1-1/\infty=1$\\
\textbf{Nondecreasing:}\\\
$\deriv{F_Y}=\dfrac{2}{y^3}>0$\\
Therefore the function satisfies all three properties and is a cdf.
\item
PDF=$f_y(y)=\begin{cases}
\dfrac{2}{y^3} \quad \text{if } y>1\\
0 \quad \text{if } y\leq 1
\end{cases}$
\item 
$z=10(y-1) \quad z/10+1=y$\\
$F_Z(z)=P(Z\leq z)=P(Y\leq z/10+1)=F_Y((z/10)+1)$
\[F_Z(z)=\begin{cases}
0 \quad \text{if } z \leq 0\\
1-\dfrac{1}{(z/10+1)^2} \quad \text{if } z>0
\end{cases}
\]
\end{enumerate}
\section*{Problem 15}
\begin{enumerate}[(a)]
\item
$F(x)=c\int_{0}^{\pi/2}\sin(x)\dx=1$\\
 $=c\big|_0^{\pi/2}-\cos(x)=c(0+1)=c$\\
 $c=1$\\
 $f(x)=\sin(x), 0<x<\pi/2$
\item
$F(x)=c\int_{-\infty}^{\infty}e^{-|x|}=1$\\
Since the integral is symmetric we can write:\\
$2c\int_{0}^{\infty}e^{-x}=1$\\
$2c\big|_0^{\infty}-e^{-x}=2c(0+1)=2c$\\
$2c=1 \quad c=1/2$\\
$f(x)=(1/2)e^{-|x|}, -\infty<x<\infty$
\end{enumerate}
\section*{Problem 16}
$P(V\leq 5)=P(T< 3)=\int_{0}^{3}\dfrac{1}{1.5}e^{-t/1.5}\ dt=\big|_0^3 -e^{-t/1.5}=-e^{-2}+1$\\
$v\geq 6$\\
$P(V\leq v)=P(2T\leq V)= P(T\leq v/2)$\\
$=\int_{0}^{v/2}\dfrac{1}{1.5}e^{-t/1.5}\ dt=\big|_0^{v/2} -e^{-t/1.5}=-e^{-v/3}+1$\\
\[F_V(v)=\begin{cases}
0 \quad -\infty<v<0\\
-e^{-2}+1 \quad 0\leq v<6\\
-e^{-v/3}+1 \quad v\geq 6
\end{cases}
\]
\end{flushleft}
\end{document}
