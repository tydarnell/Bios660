\documentclass{article}
\usepackage[utf8]{inputenc}
\usepackage[english]{babel}
\usepackage{graphicx}
\usepackage{enumerate}
\usepackage{float}
\graphicspath{ {} }
\usepackage{mathtools}
\usepackage{amsmath, amsthm, amssymb, amsfonts}
\usepackage{caption}
\usepackage{fancyhdr}
\pagestyle{fancy}
\fancyhf{}
\rhead{Ty Darnell}
\lhead{Homework 11}

% For derivatives
\newcommand{\deriv}[1]{\frac{\mathrm{d}}{\mathrm{d}x} (#1)}

% For partial derivatives
\newcommand{\pderiv}[2]{\frac{\partial}{\partial #1} (#2)}

% Integral dx
\newcommand{\dx}{\mathrm{d}x}
\allowdisplaybreaks
\begin{document}
\begin{flushleft}
\chead{Problems 1-3}
\section*{Problem 1}
\begin{align*}
\text{From} &\text{ Example 4.3.1 and Theorem 4.3.2 we know:}\\
X+Y &\sim \text{Poisson}(\theta+\lambda)\\
\text{Let } U&=X+Y, \ V=Y\\
\text{Then } X&=U-V, \ Y=V\\
f(u,v)&=\dfrac{\theta^{u-v}e^{-\theta}}{(u-v)!}\dfrac{\lambda^{v} e^{-\lambda}}{v!} \ v=0,1,\dots \ u=v,v+1, \dots\\
f(u)&=\dfrac{e^{-(\theta+\lambda)}}{u!}(\theta+\lambda)^u \ u=0,1,\dots\\
\text{Finding } &Y|X+Y\\
\text{Defining }& \text{U and V the same way:}\\
f(y|x+y)&= f(v|u)\\
f(v|u)&=\dfrac{f(u,v)}{f(u)}\\
&=\dfrac{\dfrac{\theta^{u-v}e^{-\theta}}{(u-v)!}\dfrac{\lambda^v e^{-\lambda}}{v!}}{\dfrac{e^{-(\theta+\lambda)}}{u!}(\theta+\lambda)^u}\\
&=\dfrac{u!}{(u-v)!v!}\dfrac{e^{-(\theta+\lambda)}}{e^{-(\theta+\lambda)}}\dfrac{\theta^{u-v}\lambda^v}{(\theta+\lambda)^u}\\
&={u\choose v}\dfrac{\theta^{u-v}\lambda^v}{(\theta+\lambda)^u}\\
&={u\choose v}\left(\dfrac{\lambda}{\theta+\lambda}\right)^{v}\left(\dfrac{\theta}{\theta+\lambda}\right)^{u-v}\\
\text{Which is }& \text{binomial}\left(u,\dfrac{\lambda}{\theta+\lambda}\right)\\
\text{Finding } &X|X+Y\\
\text{Define } U&=X+Y, \ V=X\\
\text{Then } X&=U-V, \ X=V\\
f_{U,V}(u,v)&=f_{X,Y}(v,u-v)=\dfrac{\theta^v e^{-\theta}}{v!} \dfrac{\lambda^{u-v} e^{-\lambda}}{(u-v)!}\\
f(u)&=\sum_{v=0}^{u}\dfrac{\theta^v e^{-\theta}}{v!}\dfrac{\lambda^{u-v}e^{-\lambda}}{(u-v)!}\\
&=\dfrac{e^{-(\theta+\lambda)}}{u!}\sum_{v=0}^{u}{u \choose v}\theta^v \lambda^{u-v}\\
\text{Using}&\text{ the binomial theorem we have:}\\
f(u)&=\dfrac{e^{-(\theta+\lambda)}}{u!}(\theta+\lambda)^u\\
f(x|x+y)&= f(v|u)\\
&=\dfrac{f(u,v)}{f(u)}\\
&=\dfrac{\dfrac{\theta^v e^{-\theta}}{v!} \dfrac{\lambda^{u-v} e^{-\lambda}}{(u-v)!}}{\dfrac{e^{-(\theta+\lambda)}}{u!}(\theta+\lambda)^u}\\
&={u \choose v}\dfrac{\theta^v \lambda^{u-v}}{(\theta+\lambda)^u}\\
&={u \choose v}\left(\dfrac{\theta}{\theta+\lambda}\right)^{v} \left(\dfrac{\lambda}{\theta+\lambda}\right)^{u-v}\\
\text{Which is }& \text{binomial}\left(u,\dfrac{\theta}{\theta+\lambda}\right)\\
\end{align*}
\section*{Problem 2}
$f_X(x)=p(1-p)^{x-1} \quad f_Y(y)=p(1-p)^{y-1}$\\
Since X and Y are independent we have:\\
$f_{X,Y}(x,y)=p(1-p)^{x-1}p(1-p)^{y-1}$\\
$=p^2(1-p)^{x+y-2}$\\
\begin{enumerate}[(a)]
\item 
Solving $V=X-Y$ for X we get $X=V+Y$\\ 
If $V>0$ then $X>Y$\\
Since $U=min(X,Y)$ this means that $U=Y$\\
Thus we have $Y=U$ and $X=U+V$
\begin{align*}
f_{U,V}(u,v)&=P(Y=u,X=u+v)\\
&=p^2(1-p)^{2u+v-2}\\
\text{Which factors to: }& (p^2(1-p)^{2u})((1-p)^{v-2})\\
\text{If } V&<0, \text{ then } X<Y\\
\text{Thus } X=U, \quad Y=U-V\\
f_{U,V}(u,v)&=P(X=u,Y=u-v)\\
&=p^2(1-p)^{2u-v-2}\\
\text{Which factors to: }& (p^2(1-p)^{2u})((1-p)^{-v-2})\\
\text{If } V&=0 \text{ then } X=Y\\
f_{U,V}(u,0)&=P(X=Y=u)=p^2(1-p)^{2u-2}\\
\text{Which factors to: }& (p^2(1-p)^{2u})((1-p)^{-2})\\
\text{Since we can factor}&\text{ all of these cases in terms of u and v, U and V are independent }
\end{align*}
\item 
\begin{align*}
\end{align*}
\item
\begin{align*}
\text{Define } T&=X+Y\\
f_{X,X+Y}(x,x+y)&=P(X=x,X+Y=t)=P(X=x,Y=t-x)=P(X=x)P(Y=t-x)\\
&=p^2(1-p)^{x-1+t-x-1}=p^2(1-p)^{t-2}
\end{align*}
\end{enumerate}
\section*{Problem 3}
\begin{enumerate}[(a)]
\item 
\begin{align*}
X_1,& X_2 \text{ are independent and distributed as:}\\
f_{X_i}(x_i)&=\dfrac{1}{\sqrt{2\pi}\sigma}e^{-x_i^2/2\sigma^2}\\
Y_1&=X_1^2+X_2^2 \quad Y_2=\dfrac{X_1}{\sqrt{Y_1}}\\
\end{align*}
\end{enumerate}



\end{flushleft}
\end{document}
