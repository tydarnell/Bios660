\documentclass{article}
\usepackage[utf8]{inputenc}
\usepackage[english]{babel}
\usepackage{graphicx}
\usepackage{enumerate}
\usepackage{float}
\graphicspath{ {} }
\usepackage{mathtools}
\usepackage{amsmath, amsthm, amssymb, amsfonts}
\usepackage{caption}
\usepackage{fancyhdr}
\pagestyle{fancy}
\fancyhf{}
\rhead{Ty Darnell}
\lhead{Homework 4}

% For derivatives
\newcommand{\deriv}[1]{\frac{\mathrm{d}}{\mathrm{d}x} (#1)}

% For partial derivatives
\newcommand{\pderiv}[2]{\frac{\partial}{\partial #1} (#2)}

% Integral dx
\newcommand{\dx}{\mathrm{d}x}
\begin{document}
\begin{flushleft}
\chead{Problems 1-3}
\section*{Problem 1}
In order for the sum to be greater than or equal to $k$, we need at least $k$ ones. The number of ways to get $k$ ones is:
\[\left(\sum \limits_{i=1}^{n}x_i=k\right) \to {n \choose k}
\]
Then we have to take in account all of the ways to get $k+1,k+2,...,n$ ones.\\
Summing all of these possibilities gives us:
\[\sum \limits_{i=k}^{n}{n \choose i}
\]
\section*{Problem 2}
\begin{align*}
\text{We want to show } \sum \limits_{i=0}^{n}(-1)^i{n \choose i}&=0 \text{ for } n>0 \\
\text{The binomial}& \text{ theorem states:}\\
(x+y)^n&=\sum \limits_{k=0}^{n}{n \choose k}x^k y^{n-k} \quad \forall \ n\in \mathbb{N}  \label{1}\tag{1}\\
\text{Let }& x=-1, y=1, k=i\\
\text{Then } \eqref{1} \text{ becomes }& \sum \limits_{i=0}^{n}{n \choose i}(-1)^i(1)^{n-i}=(-1+1)^n\\
\text{Then we have } \sum \limits_{i=0}^{n}{n \choose i}(-1)^i(1)&=0^n\\
=\sum \limits_{i=0}^{n}{n \choose i}(-1)^i&=0\\
\text{Therefore }
\sum \limits_{i=0}^{n}{n \choose i}(-1)^{i}&=0 \text{ for } n>0
\end{align*}
\pagebreak
\section*{Problem 3}
\begin{enumerate}[(a)]
\item $\dfrac{4{13\choose 5}}{{52\choose 5}} \approx .001980$ \\
\item $\dfrac{13{4\choose 2}{12 \choose 3}4^3}{{52\choose 5}} \approx .422569$\\
\item $\dfrac{{13 \choose 2}{4\choose 2}{4\choose 2}11*4}{{52\choose 5}}\approx .047539$\\
\item $\dfrac{13{4\choose 3}{12 \choose 2}*4^2}{{52\choose 5}}\approx .021128$\\
\item $\dfrac{13*12*4}{{52\choose 5}}\approx .000240$
\end{enumerate}
\pagebreak
\section*{Problem 4}
\chead{Problems 4-6}
\[\dfrac{16*4}{52*51}+\dfrac{4*16}{52*51} = \dfrac{16*4*2}{52*51} \approx .048265
\]
\section*{Problem 5}

\end{flushleft}
\end{document}
