\documentclass{article}
\usepackage[utf8]{inputenc}
\usepackage[english]{babel}
\usepackage{graphicx}
\usepackage{enumerate}
\usepackage{float}
\graphicspath{ {} }
\usepackage{mathtools}
\usepackage{amsmath, amsthm, amssymb, amsfonts}
\usepackage{caption}
\usepackage{fancyhdr}
\pagestyle{fancy}
\fancyhf{}
\rhead{Ty Darnell}
\lhead{Final Notes 2}

% For derivatives
\newcommand{\deriv}[1]{\frac{\mathrm{d}}{\mathrm{d}x} (#1)}

% For partial derivatives
\newcommand{\pderiv}[2]{\frac{\partial}{\partial #1} (#2)}

% Integral dx
\newcommand{\dx}{\mathrm{d}x}
\allowdisplaybreaks
\begin{document}
\begin{flushleft}
\section*{Homework 9}
\subsection*{Gamma Distribution}
Gamma Function: $\Gamma(\alpha)=\int_{0}^{\infty}t^{\alpha-1}e^{-t} \ dt$\\
$\Gamma(\alpha+1)=\alpha \Gamma(\alpha) \ \alpha>0$\\
$\Gamma(n)=(n-1)! \quad n \in \mathbb{Z}$\\
$\Gamma(1/2)=\sqrt{\pi}$\\
$f(x|\alpha,\beta)=\dfrac{1}{\Gamma(\alpha) \beta^\alpha}x^{\alpha-1}e^{-x/\beta}$\\
$\alpha$ is the shape parameter, influences the peakedness of the distribution\\
$\beta$ is the scale parameter, influences the spread of the distribution\\
$EX^v=\dfrac{\beta^v \Gamma(v+\alpha)}{\Gamma(\alpha)}$\\
$\Gamma(\alpha+v)=\int_{0}^{\infty}x^{v+\alpha-1}e^{-x} \ \dx$\\
$EX=\alpha \beta$\\
$\int_{0}^{\infty}e^{-x^2/2} \ dz=\dfrac{\sqrt{2\pi}}{2}=\sqrt{\dfrac{\pi}{2}}$\\
$\int_{0}^{\infty}x^2e^{-x^2}$ is the same\\
\subsection*{Beta Distribution}
$f(x|\alpha,\beta)=\dfrac{1}{B(\alpha,\beta)}x^{\alpha-1}(1-x)^{\beta-1}$\\
Beta Function: $B(\alpha,\beta)=\int_{0}^{1}x^{\alpha-1}(1-x)^{\beta-1} \ \dx$ \\ 
$B(\alpha,\beta)=\dfrac{\Gamma(\alpha)\Gamma(\beta)}{\Gamma(\alpha+\beta)}$\\
$EX^n=\dfrac{B(\alpha+n,\beta)}{B(\alpha,\beta)}=\dfrac{\Gamma(\alpha+n)\Gamma(\alpha+\beta)}{\Gamma(\alpha+\beta+n)\Gamma(\alpha)}$\\
\subsection*{Mgfs}
Negative Binomial Mgf $\left(\dfrac{p}{1-(1-p)e^t}\right)^r$\\
\subsection*{Exponential Families}
A family of pdfs or pmfs is called an exponential family if it can be expressed as
\[f(x|\theta)=h(x)c(\theta)\exp\left(\sum_{i=1}^{k}w_i(\theta)t_i(x)\right)
\]
\end{flushleft}
\end{document}
