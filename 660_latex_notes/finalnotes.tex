\documentclass{article}
\usepackage[utf8]{inputenc}
\usepackage[english]{babel}
\usepackage{graphicx}
\usepackage{enumerate}
\usepackage{float}
\graphicspath{ {} }
\usepackage{mathtools}
\usepackage{amsmath, amsthm, amssymb, amsfonts}
\usepackage{caption}
\usepackage{fancyhdr}
\pagestyle{fancy}
\fancyhf{}
\rhead{Ty Darnell}
\lhead{Final Notes}

% For derivatives
\newcommand{\deriv}[1]{\frac{\mathrm{d}}{\mathrm{d}x} (#1)}

% For partial derivatives
\newcommand{\pderiv}[2]{\frac{\partial}{\partial #1} (#2)}

% Integral dx
\newcommand{\dx}{\mathrm{d}x}
\allowdisplaybreaks
\begin{document}
\begin{flushleft}
\chead{Notes 14}
\section*{Notes 14}
\subsection*{Convolution}
If X and Y are independent continuous r.v.s with pdfs $f_X(x)$ and $f_Y(y)$, then the pdf of $Z=X+Y$ is:
\[f_Z(z)=\int_{-\infty}^{\infty}f_X(w)f_Y(z-w)\ dw
\]
\subsection*{Sum of Two Independent Poissons}
$X\sim Pois(\lambda_1), \ Y\sim Pois(\lambda_2)$\\
$U=X+Y \ V=Y$\\
$X=U-V \ Y=V$\\
Joint PMF of U and V is:\\
$f_{U,V}(u,v)=f_{X,Y}(u-v,v)=\dfrac{e^{-\lambda_1}\lambda_1^{u-v}}{(u-v)!}\dfrac{e^{-\lambda_2}\lambda_2^v}{v!}$\\
The distribution of $U=X+Y$ is the marginal:\\
$f_U(u)=\sum_{v=0}^{u}\dfrac{e^{-\lambda_1}\lambda_1^{u-v}}{(u-v)!}\dfrac{e^{-\lambda_2}\lambda_2^v}{v!}$\\
$=\dfrac{e^{-(\lambda_1+\lambda_2)}}{u!}\sum_{v=0}^{u} {u \choose v}\lambda_1^{u-v}\lambda_2^v$\medbreak
Because of the binomial theorem\medbreak
$=\dfrac{e^{-(\lambda_1+\lambda_2)}}{u!}(\lambda_1+\lambda_2)^u$\medbreak
$U\sim Pois(\lambda_1+\lambda_2)$
\subsection*{Jacobian}
$J(u,v)$ is the Jacobian of the transformation $(x,y)\to (u,v)$ given by:\medbreak
$J(u,v)=\dfrac{\partial(x,y)}{\partial(u,v)}=
\begin{bmatrix}
\dfrac{\partial x}{\partial u}& \dfrac{\partial x}{\partial v}\\
\dfrac{\partial y}{\partial u}& \dfrac{\partial y}{\partial v}\\
\end{bmatrix}$
\subsection*{Functions of Independent Random Variables}
Let X and Y be independent r.v.s\\
Let $g:\mathbb{R}\to \mathbb{R}$ and $h:\mathbb{R}\to \mathbb{R}$ be functions\\
Then the r.v.s $U=g(X)$ and $V=h(Y)$ are independent
\subsection*{Ratio of Two Independent Normals}
Let $X \sim N(0,1)$ and $Y\sim N(0,1)$\\
The ratio $X/Y$ has the Cauchy distribution\\
Let $U=X/Y$ and $V=Y$ \quad Then $X=UV$ and $Y=V$ \quad $J(u,v)=v$\\
$f_{X,Y}(x,y)=\dfrac{1}{\sqrt{2\pi}}e^{-x^2/2}\dfrac{1}{\sqrt{2\pi}}e^{-y^2/2}=\dfrac{1}{2\pi}e^{-(x^2+y^2)/2}$\\
$f_{U,V}(uv,v)=\dfrac{1}{2\pi}e^{-[(uv)^2+v^2]/2}*|v|=\dfrac{|v|}{2\pi}e^{-(u^2+1)v^2/2}$\\
$f_{U}(u)=\int_{-\infty}^{\infty}f_{UV}(u,v) \ dv=2\int_{0}^{\infty}\dfrac{v}{2\pi}e^{-(u^2+1)v^2/2} \ dv$\\
$=\dfrac{1}{\pi}\int_{0}^{\infty}e^{-(u^2+1)z} \ dz=\dfrac{1}{\pi(u^2+1)}$
\subsection*{Sum of Two Independent Random Variables}
Suppose X and Y are independent, find distribution of $Z=X+Y$\\
In general: $F_Z(z)=P(X+Y\leq z)=P(\{(x,y) \text{ such that } x+y\leq z\})$\\
Approaches:
\begin{itemize}
\item Bivariate transformation method (continuous and discrete)\\
\item Discrete convolution:\\
$f_Z(z)=\sum_{x+y=z}f_X(x)f_Y(y)=\sum_{x}f_X(x)f_Y(z-x)$\\
\item Continuous convolution\\
\item Mgf/cf method (continuous and discrete)\\
$\phi_Z{\theta}=\phi_X(\theta)\phi_Y(\theta)$\\
$Z=X-Y \quad \phi_Z{\theta}=\phi_X(\theta)\phi_Y(-\theta)$
\end{itemize}
\section*{Notes 15}
\chead{Notes 15}
\subsection*{Conditional Expectation and Variance}
\textbf{Iterative Expectation Formula}\\
$EX=E(E(X|Y))$\\
\textbf{Variance}\\
$Var[g(Y)]=E[g(y)-E(g(Y))]^2$\medbreak
$VarX=E(Var(X|Y))+Var(E(X|Y))$\\
$Var(g(Y)|X)=E\{[g(Y)-E(g(Y)|X)]^2|X\}$\\
where both expectations are taken with respect to $f_{Y||X}(y)$\\
\begin{itemize}
\item $E(Var(X|Y))=E\{[X-E(X|Y)]^2\}$\\
\item $Var(E(X|Y))=E\{[E(X|Y)-EX]^2\}$\\
\end{itemize}
\subsection*{Covariance and Correlation}
$Cov(X,Y)=E[(X-\mu_X)(Y-\mu_Y)]=\sigma_{XY}$\\
Correlation =$\rho_{XY}=\dfrac{Cov(X,Y)}{\sqrt{VarX \ VarY}}=\dfrac{\sigma_{XY}}{\sigma_X \sigma_Y}$\\
$=E\left[ \left( \dfrac{X-\mu_X}{\sigma_X} \right) \left( \dfrac{Y-\mu_Y}{\sigma_Y} \right) \right]$\medbreak
X and Y are uncorrelated iff:\\
$Cov(X,Y)=0$ or equivalently $\rho_{XY}=0$\\
$Cov(X,Y)=E(XY)-E(X)E(Y)$\\
If X and Y are independent and $Cov(X,Y)$ exists, then $Cov(X,Y)=0$\\
If X and Y are uncorrelated this does not imply independence.\\
\subsection*{Linear Combinations}
$Cov(aX+B_Y,Z)=aCov(X,Z)+bCov(Y,Z)$\medbreak
$Var(aX+bY)=a^2Var(X)+b^2Var(Y)+2abCov(X,Y)$\medbreak
$Corr(aX+b,cY+d)=\dfrac{ac}{|ac|}Corr(X,Y)$

\subsection*{Standard Bivariate Normal}
$f_{XY}(x,y)=\dfrac{1}{2\pi\sqrt{1-\rho^2}}\exp{\left[-\dfrac{x^2-2\rho xy+y^2}{2(1-\rho^2)}\right]}$\\
Both X and Y have marginal distributions are $N(0,1)$\\
Correlation of X and Y is $\rho$\\
Conditional Distribution are normal:\\
$Y|X \sim N(\rho X,1-\rho^2) \quad X|Y \sim N(\rho Y, 1-\rho^2)$\\
The means are the regression lines of Y on X and X on Y respectively.
\subsection*{Bivariate Normal}
Let $\tilde{X}$ and $\tilde{Y}$ have a standard bivariate normal distribution with correlation $\rho$\\
Let $X=\mu_X+\sigma_X\tilde{X} \quad Y=\mu_Y+\sigma_Y\tilde{Y}$\\
Then $(X,Y)$ has the bivariate normal density:\\
$f_{XY}(x,y)=\left(2\pi \sigma_X \sigma_Y \sqrt{1-\rho^2} \right)^{-1}\exp{\left\{-\dfrac{1}{2(1-\rho^2)}\left[\left(\dfrac{x-\mu_X}{\sigma_X} \right)^2- 2\rho \left(\dfrac{x-\mu_X}{\sigma_X} \right)\left(\dfrac{y-\mu_Y}{\sigma_Y} \right)+\left(\dfrac{y-\mu_Y}{\sigma_Y} \right)^2 \right] \right\}}$\medbreak
Marginal distributions: $N(\mu_X,\sigma^2_X) \quad N(\mu_Y,\sigma^2_Y)$\\
$Corr(X,Y)=\rho$\medbreak
Conditional distributions are normal:\\
$Y|X \sim N[\mu_Y+\rho(\sigma_Y/\sigma_X)(x-\mu_X),\sigma^2_Y(1-\rho^2)]$\medbreak
Distribution of $aX+bY$ is:\\
$N(a\mu_X+b\mu_Y,a^2\sigma^2_X+b^2\sigma^2_Y+2ab\rho \sigma_X \sigma_Y)$
\subsection*{Multivariate Distributions}
$\boldsymbol{X}=(X_1,X_2,\dots, X_n)$\\
If $\boldsymbol{X}$ is discrete then:\\
$P(\boldsymbol{X} \in A)=\sum_{\boldsymbol{X} \in A} f(\boldsymbol{X})$\\
where $f(\boldsymbol{X})$ is the joint pmf\\
If $\boldsymbol{X}$ is continuous then:\\
$P(\boldsymbol{X}\in A)=\int \cdots \int_{A}f(x_1,\dots,x_n)dx_1,\dots dx_n$\medbreak
\subsection*{Marginals and Conditionals}
The \textbf{marginal} pdf or pmf of any subset of coordinates is found by integrating or summing the joint pdf or pmf over all possible values of the other coordinates.\medbreak

The \textbf{conditional} pdf or pmf of a subset of coordinates given the values of the remaining coordinates is found by dividing the full joint pdf or pmf by the joint pdf or pmf of the remaining variables.

\subsection*{Multivariate Independence}
Independent Random Vectors:\\
Let $\boldsymbol{X_1},\dots,\boldsymbol{X_n}$ be random vectors with joint pdf or pmf $f(\boldsymbol{X_1},\dots,\boldsymbol{X_n})$\\
Let $f\boldsymbol{X_j}(\boldsymbol{x_j})$ be the marginal pdf or pmf of $\boldsymbol{X_j}$.\\
Then $\boldsymbol{X_1},\dots,\boldsymbol{X_n}$ are \textbf{mutually independent} random vectors if:\\
$\forall$ $(\boldsymbol{X_1},\dots,\boldsymbol{X_n})$: \quad  $f(\boldsymbol{X_1},\dots,\boldsymbol{X_n})=\prod_{j=1}^{n}f\boldsymbol{X_j}(\boldsymbol{x_j})$

\section*{Notes 16}
\chead{Notes 16}
\subsection*{Inequalities}
\textbf{Chebychev Inequality}\\
$p[g(X)\geq r]leq \dfrac{E[g(X)]}{r}$\\
If X is nonnegative and g is a positive non-decreasing function then:\\
$P\{X\geq a \}\leq \dfrac{E[g(X)]}{g(a)}$\\
\textbf{Special Cases:}\\
$X\geq 0$ \quad $P\{X\geq a \}\leq \dfrac{E(e^{tX})}{e^{ta}}$\\
$L^p$ Space- consists of all r.v.s whose $p^{th}$ absolute power is integrable, $E(|X|^p)<\infty$\\
\textbf{Triangle Inequality}\\
$|a+b|\leq |a|+|b|$\\
\textbf{Convex Functions}\\
A function $g: I\to R$ is convex for any $\lambda \in [0,1]$ and any points x and y in I\\
$g[\lambda x+(1-\lambda)y]\leq \lambda g(x)+(1-\lambda)g(y)$\\
A differentiable function g is convex iff it lays above all tangents.\\
A twice differentiable function g is convex iff its second derivative is non-negative\\
concave if -g is convex on I\\
\textbf{Jensen's Inequality}\\
Let $X \in L^1$ and $g(x)$ be a convex function where $E[g(X)]$ exists. Then:\\
$E[g(X)]\geq g[EX]$\\
with equality iff for every line $a+bx$ tangent to $g(x)$ at $x=EX, P[g(X)=a+bX]=1$\\
direction of inequality is reversed if g is concave\\
\textbf{Young's Inequality}\\
Let $a,b >0$ and $p,q>1$ with $1/p+1/q=1$ Then:\\
$\dfrac{a^p}{p}+\dfrac{b^q}{q}\geq ab$\\
with equality iff $a^p=b^q$\\
\textbf{Holder's Inequality}\\
Suppose $X \in L^p, Y \in L^q$ where $p,q>1$ and $1/p+1/q=1$ Then:\\
$E[|XY|]\leq [E|X|^p]^{1/p}E[|Y|^q]^{1/q}$\\
with equality if $X^p=cY^q$ for some $c \in \mathbb{R}$\\
\textbf{Cauchy-Schwartz Inequality}\\
corollary of Holders where $p=q=2$\\
$E[|XY|]\leq [E|X|^2]^{1/2}E[|Y|^2]^{1/2}=\sqrt{E[X^2]E[Y^2]}$\\
with equality if $X=cY$\\
\textbf{Lyapunov's Inequality}\\
corrallary of Holders\\
for $1\leq r\leq s$ and $X\in L^s$\\
$E[|X|^r]^{1/r}\leq E[|X|^s]^{1/s}$\\
\textbf{Application of Cauchy-Schwartz}\\
$p=\dfrac{Cov(X,Y)}{\sqrt{Var(X)Var(Y)}}$\\
Then $|p|\leq 1$ with equality iff $Y-\mu_Y=c(X-\mu_X)$\\
\textbf{Minkowski's Inequality}\\
Suppose $X,Y \in L^p, p\geq 1$ Then $(X+Y)\in L^p$ and\\
$[E|X+Y|^p]^{1/p}\leq [E|X|^p]^{1/p}+[E|Y|^p]^{1/p}$\\
\subsection*{Order Statistics}
\textbf{Distribution of the Maximum}\\
The cdf of $Z=max(Y_1,\dots,Y_n)$ is\\
\begin{align*}
F_Z(z)&= P\{Z\leq z \}\\
&=P\{Y_1\leq z, Y_2 \leq z,\dots, Y_n \leq z \}\\
&=\prod_{j=1}^{n}P\{Y_j\leq z \} \text{ indep}\\
&=F_Y(z)^n \text{ ident. distrib}
\end{align*}
Thus the pmf is:\\
$f_Z(z)=nF_Y(z)^{n-1}f_Y(z)$\\
\textbf{Distribution of the Minimum}\\
$W=min(Y_1,\dots,Y_n)$\\
$F_W(w)=1-(1-F_Y(w))^n$\\
$f_W(w)=n(1-F_Y(w))^{n-1}f_Y(w)$\\
\textbf{Order Statistics}\\
Let $Y_1,Y_2,\dots, Y_n$ be iid with pdf $f_Y(x)$\\
Order the observations:\\
$Y_{(1)}\leq Y_{(2)}\leq \cdots \leq Y_{(n)}$\\
The $Y_{(i)}$ are called order statistics. Minimum is $Y_{(1)}$ max is $Y_{(n)}$\\
We are interested in finding the distribution of an arbitrary $Y_{(i)}$ as well as the joint distributions of sets of $Y_{(i)}$s and $Y_{(j)}$s\\
ex: Range=$Y_{(n)}-Y_{(1)}$\\
$r^{th}$\textbf{ order statistic}\\
We need to find the density of $Y_{(r)}$ at a value y\\
Consider 3 intervals $(-\infty,y),[y,y+dy),[y+dy,\infty)$\\
The number of observations in each of the intervals follows the tri-nomial distribution:\\
$f(s_1,s_2,s_3)=\dfrac{n!}{s_1!s_2!s_3!}p_1^{s_1}p_2^{s_2}p_3^{s_3}$\\
The event that $y\leq Y_{(r)}<y+dy$ is the event we have:\\
$(r-1)$ observations are less than y,\\
$(n-r)$ observations are greater than y\\
1 observation is in interval $y,y+dy$\\
In the trinomial distribution this corresponds to:\\
$s_1=r-1,s_2=1,s_3=n-r$\\
$p_1=F_Y(y), p_2=f_Y(y)dy, p_3=1-F_Y(y+dy)$\\
\textbf{Theorem 5.4.6}\\
Let $X_{(1)},\dots,X_{(n)}$ denote the order statistics of a random sample, $X_1,\dots, X_n$ from a continuous population with cdf $F_X(x)$ and pdf $f_X(x)$. Then the joint pdf of $X_{(i)}$ and $X_{(j)}$, $1\leq i <j\leq n$, is:
\[f_{X_{(i)},X_{(j)}}(u,v)=\dfrac{n!}{(i-1)!(j-1-i)!(n-j)!}f_X(u)f_X(v)[F_X(u)]^{i-1}[F_X(v)-F_X(u)]^{j-1-i}[1-F_X(v)]^{n-j}\]
for $-\infty<u,v<\infty$\\
The joint pdf of all the order statistics is:
\[f_{X_{(1)}\dots,X_{(n)}}(x_1,\dots,x_n)=\begin{cases}
n!f_X(x_1)\cdots f_X(x_n) \quad -\infty<x_1<\cdots<x_n<\infty\\
0 \quad \text{otherwise}
\end{cases}
\]
\end{flushleft}
\end{document}
