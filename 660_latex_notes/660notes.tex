\documentclass[openany]{book}
\usepackage[utf8]{inputenc}
\title{Bios 660 Notes}
\author{Ty Darnell}
\date{ }
\usepackage[english]{babel}
 \usepackage{graphicx}
 \usepackage{float}
 \graphicspath{ {} }
 \usepackage{mathtools}
 \usepackage{amsmath, amsthm, amssymb, amsfonts}
 \usepackage{caption}
 \usepackage{titlepic}
\usepackage{hyperref}
\hypersetup{
    colorlinks=true,
    linkcolor=blue,
    filecolor=magenta,      
    urlcolor=cyan,
    pdftitle={Bios 660 Notes},
    pdfauthor={Ty Darnell},
    bookmarksopen=true,
}
\numberwithin{equation}{section}
\newtheorem{proposition}{Proposition}[section]
\begin{document}
\tableofcontents
\begin{flushleft}
\chapter{Elementary Set Theory}
\section{Basic Notation}
\subsection{Common Notation}
$\left\{w\right\}$ denotes a set\medbreak
$w$ denotes an element \medbreak
$(a,b)$ is an open interval, not including a and b \medbreak
$[a,b]$ is a closed interval, including a and b \medbreak
$\left\{w: \ \text{a statment}\right\}$: the set of elements w for which the statement holds.\\
ex: $\left\{w:a<w<b\right\}$
\subsection{The Sample Space}
\textbf{Sample Space}: denoted by $\Omega$, as a non-empty set of all the elements concerned. These elements are called points
and are denoted with lower case letters. \medbreak
\section{Set Operations}
\textbf{Difference}: $A-B=\left\{w:w\in A, w \notin B \right\}=A\cap B^c$ \medbreak
\textbf{Symmetric Difference}: $A\Delta B=(A-B)\cup (B-A)=\left\{w:w\in \text{exactly one of A and B} \right\}$ \medbreak
\textbf{Disjoint}: Two sets are disjoint if $A \cap B=\emptyset$ \medbreak
\textbf{Disjoint Union}: For two disjoint sets the disjoint union is $A \cup B=A+B$
\textbf{Infimum}(inf): greatest lower bound \medbreak
\textbf{Supremum}(sup): least upper bound \medbreak



\end{flushleft}
\end{document}
